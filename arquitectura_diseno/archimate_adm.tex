\chapter{Lenguaje y Método}
\section{Introducción}
contenido...
\newpage
\section{Archimate}
contenido...  
\begin{itemize}
	\item capítulo 2
	\item tablas de resumen capítulos 3,4,5,6,10
\end{itemize}

\newpage
\section{Tabla de Negocio}
\begin{center}
	\begin{tabular}{|c | p{5cm} |  p{5cm}|}
		
		\hline
		Concepto         & Descripción                                                            & Notación
		\\ \hline
		Actor de negocio & Una entidad organizacional que es capaz de realizar un comportamiento. &  
		
		\raisebox{-\totalheight}{\includegraphics[width=0.3\textwidth]{arquitectura_diseno/imgs/ADMneg1.png} }
		\\ \hline
		Rol de negocio
		&                  
		La responsabilidad de realizar un comportamiento específico, al que se le puede asignar un actor.                                                    &  
		\raisebox{-\totalheight}{\includegraphics[width=0.3\textwidth]{arquitectura_diseno/imgs/ADMneg2.png} }
		
		\\ \hline
		Colaboración de negocio
		&                  
		
		Un agregado de dos o más negocios
		roles que trabajan juntos para realizar un comportamiento colectivo.                                                    &  
		\raisebox{-\totalheight}{\includegraphics[width=0.3\textwidth]{arquitectura_diseno/imgs/ADMneg3.png} }			 
		
		\\ \hline
		Interfaz de negocio
		&                  
		
		Un punto de acceso donde un servicio comercial está disponible para el medio ambiente.                                                  &  
		\raisebox{-\totalheight}{\includegraphics[width=0.3\textwidth]{arquitectura_diseno/imgs/ADMneg4.png} }	
		
		\\ \hline
		Ubicación
		&   Un punto o extensión conceptual en el espacio.                                                                     & \raisebox{-\totalheight}{\includegraphics[width=0.3\textwidth]{arquitectura_diseno/imgs/ADMneg5.png} }
		\\ \hline
		Objeto de negocio
		&      
		Un elemento pasivo que tiene relevancia desde una perspectiva empresarial.                                                                  & 
		\raisebox{-\totalheight}{\includegraphics[width=0.3\textwidth]{arquitectura_diseno/imgs/ADMneg11.png} }
		\\ \hline
		
		Proceso de negocio
		&      
		Un elemento de comportamiento que agrupa el comportamiento según un orden de actividades. Su objetivo es producir un conjunto definido de productos o servicios comerciales.                                                                &
		\raisebox{-\totalheight}{\includegraphics[width=0.3\textwidth]{arquitectura_diseno/imgs/ADMneg6.png} } 
		\\ \hline
		
		Función de negocio
		&      
		
		Un elemento de comportamiento que agrupa el comportamiento en función de un conjunto de criterios elegidos (por lo general, se requieren recursos comerciales y / o competencias).                                                               &
		\raisebox{-\totalheight}{\includegraphics[width=0.3\textwidth]{arquitectura_diseno/imgs/ADMneg7.png} } 
		
		\\ \hline
	\end{tabular}

\begin{tabular}{|c | p{5cm} |  p{5cm}|}
\\ \hline
		Interacción del negocio
		&      
		Un elemento de comportamiento que describe el comportamiento de una colaboración empresarial.                                                              &
		\raisebox{-\totalheight}{\includegraphics[width=0.3\textwidth]{arquitectura_diseno/imgs/ADMneg8.png} } 
		\\ \hline		
		Evento del negocio
		&      
		Algo que sucede (internamente o
		externamente) e influye en el comportamiento.                                                        &
		\raisebox{-\totalheight}{\includegraphics[width=0.3\textwidth]{arquitectura_diseno/imgs/ADMneg9.png} } 
		
		\\ \hline		
		Servicio del negocio
		&      
		Un servicio que satisface las necesidades comerciales de un cliente (interno o externo a la organización).                                                      &
		\raisebox{-\totalheight}{\includegraphics[width=0.3\textwidth]{arquitectura_diseno/imgs/ADMneg10.png} } 		
		
		\\ \hline		
		Representación
		&      
		
		Una forma perceptible de la información transportada por un objeto comercial.                                                     &
		\raisebox{-\totalheight}{\includegraphics[width=0.3\textwidth]{arquitectura_diseno/imgs/ADMneg12.png} } 
		
		\\ \hline		
		Significado
		&      
		
		
		El conocimiento o experiencia presente en un objeto de negocio o su representación, dado un contexto particular.                                                    &
		\raisebox{-\totalheight}{\includegraphics[width=0.3\textwidth]{arquitectura_diseno/imgs/ADMneg13.png} }
		
		\\ \hline		
		Valor
		&      
		
		
		
		El valor relativo, la utilidad o la importancia de un servicio o producto comercial.                                                   &
		\raisebox{-\totalheight}{\includegraphics[width=0.3\textwidth]{arquitectura_diseno/imgs/ADMneg14.png} }	
		
		\\ \hline		
		Producto
		&      
		
		Una colección coherente de servicios,
		acompañado de un contrato / conjunto de acuerdos, que se ofrece en conjunto a clientes (internos o externos).                                        &
		\raisebox{-\totalheight}{\includegraphics[width=0.3\textwidth]{arquitectura_diseno/imgs/ADMneg15.png} }	
		
		\\ \hline		
		Contrato
		&      
		Una especificación formal o informal de acuerdo que especifica los derechos y obligaciones asociados con un producto..                                        &
		\raisebox{-\totalheight}{\includegraphics[width=0.3\textwidth]{arquitectura_diseno/imgs/ADMneg16.png} }	
		
		\\ \hline					
	\end{tabular}
\end{center}


\section{Tabla de Aplicación}

\begin{center}
	\begin{tabular}{|c | p{5cm} |  p{5cm}|}
		
		\hline
		Concepto         & Descripción                                                            & Notación
		\\ \hline
		
		Componente de aplicación
		&      
		Una parte modular, implementable y reemplazable de un sistema de software que encapsula su comportamiento y datos y los expone a través de un conjunto de interfaces.                                       &
		\raisebox{-\totalheight}{\includegraphics[width=0.3\textwidth]{arquitectura_diseno/imgs/ADMap1.png} }	
		
		\\ \hline		
		
		Colaboración de aplicación
		&      
		
		Un agregado de dos o más componentes de aplicación que trabajan juntos para realizar un comportamiento colectivo.                                      &
		\raisebox{-\totalheight}{\includegraphics[width=0.3\textwidth]{arquitectura_diseno/imgs/ADMap2.png} }	
		
		\\ \hline		
		
		Interfaz de aplicación
		&      
		
		Un punto de acceso donde un servicio de aplicación está disponible para un usuario u otro componente de aplicación.                                 &
		\raisebox{-\totalheight}{\includegraphics[width=0.3\textwidth]{arquitectura_diseno/imgs/ADMap3.png} }	
		
		\\ \hline		
		
		Objeto de Datos
		&      
		
		
		Un elemento pasivo adecuado para el procesamiento automatizado.                              &
		\raisebox{-\totalheight}{\includegraphics[width=0.3\textwidth]{arquitectura_diseno/imgs/ADMap7.png} }	
		
		
		\\ \hline		
		
		Función de aplicación
		&      
		Un elemento de comportamiento que agrupa el comportamiento automatizado que puede realizar un componente de la aplicación.                             &
		\raisebox{-\totalheight}{\includegraphics[width=0.3\textwidth]{arquitectura_diseno/imgs/ADMap4.png} }	
		
		\\ \hline		
		
		Interacción de aplicación
		&      
		Un elemento de comportamiento que describe el comportamiento de una colaboración de aplicaciones.                          &
		\raisebox{-\totalheight}{\includegraphics[width=0.3\textwidth]{arquitectura_diseno/imgs/ADMap5.png} }	
		
		\\ \hline		
		
		Servicio de aplicación
		&      
		Un servicio que expone el comportamiento automatizado.                    &
		\raisebox{-\totalheight}{\includegraphics[width=0.3\textwidth]{arquitectura_diseno/imgs/ADMap6.png} }	
		
		\\ \hline		
	\end{tabular}
\end{center}


\section{Tabla de Tecnología}

\begin{center}
	\begin{tabular}{|p{5cm} | p{5cm} |  p{5cm}|}
		
		\hline
		Concepto         & Descripción                                                            & Notación
		\\ \hline
		Nodo & 
		Un recurso computacional sobre el cual los artefactos pueden ser almacenados o desplegados para
		ejecución.&  
		
		\raisebox{-\totalheight}{\includegraphics[width=0.3\textwidth]{arquitectura_diseno/imgs/ADMte1.png} }
		\\ \hline
		
		Equipo & 
		Un recurso de hardware sobre el que se pueden almacenar o implementar artefactos para su ejecución.&  
		
		\raisebox{-\totalheight}{\includegraphics[width=0.3\textwidth]{arquitectura_diseno/imgs/ADMte2.png} }
		\\ \hline
		
		Red & 
		
		Un medio de comunicación entre dos o más dispositivos.&  
		
		\raisebox{-\totalheight}{\includegraphics[width=0.3\textwidth]{arquitectura_diseno/imgs/ADMte3.png} }
		\\ \hline
		
		Ruta de comunicación & 
		
		Un enlace entre dos o más nodos, a través del cual estos nodos pueden intercambiar datos.&  
		
		\raisebox{-\totalheight}{\includegraphics[width=0.3\textwidth]{arquitectura_diseno/imgs/ADMte4.png} }
		\\ \hline
		
		Interfaz de infraestructura & 
		
		Un punto de acceso donde los servicios de infraestructura ofrecidos por un nodo pueden ser accedidos por otros nodos y componentes de la aplicación.&  
		
		\raisebox{-\totalheight}{\includegraphics[width=0.3\textwidth]{arquitectura_diseno/imgs/ADMte5.png} }
		\\ \hline
		
		
		Sistema de software & 
		
		Un entorno de software para tipos específicos de componentes y objetos que se implementan en él en forma de artefactos.&  
		
		\raisebox{-\totalheight}{\includegraphics[width=0.3\textwidth]{arquitectura_diseno/imgs/ADMte6.png} }
		\\ \hline
		
		Función de infraestructura & 
		
		Un elemento de comportamiento que agrupa el comportamiento infraestructural que puede realizar un nodo.&  
		
		\raisebox{-\totalheight}{\includegraphics[width=0.3\textwidth]{arquitectura_diseno/imgs/ADMte7.jpg} }
		\\ \hline
		
		Servicio de infraestructura & 
		
		Una unidad de funcionalidad externamente visible, proporcionada por uno o más nodos, expuesta a través de interfaces bien definidas, y
		significativo para el medio ambiente.&  
		
		\raisebox{-\totalheight}{\includegraphics[width=0.3\textwidth]{arquitectura_diseno/imgs/ADMte8.png} }
		
	\end{tabular}
	
	
	\begin{tabular}{|p{5cm} | p{5cm} |  p{5cm}|}
		
		\\ \hline
		
		Artefacto & 
		
		Una pieza física de datos que se usa o
		producido en un proceso de desarrollo de software,
		o por despliegue y operación de un sistema.&  
		
		\raisebox{-\totalheight}{\includegraphics[width=0.3\textwidth]{arquitectura_diseno/imgs/ADMte9.png} }
		\\ \hline
		
	\end{tabular}
\end{center}
\newpage
\section{Tabla de Relaciones}

\begin{center}
	\begin{tabular}{|c | p{5cm} |  p{5cm}|}
		
		\hline
		Relaciones estructurales       && Notación
		\\ \hline
		Asociación & 
		La asociación modela una relación entre objetos que no está cubierta por otra relación más específica.&  
		
		\raisebox{-\totalheight}{\includegraphics[width=0.3\textwidth]{arquitectura_diseno/imgs/ADMre1.png} }
		\\ \hline
		
		Acceso & 
		La relación de acceso modela el acceso de conceptos de comportamiento a objetos comerciales o de datos.&  
		
		\raisebox{-\totalheight}{\includegraphics[width=0.3\textwidth]{arquitectura_diseno/imgs/ADMre2.png} }
		\\ \hline		
		
		Usado por & 
		
		La relación utilizada modela el uso de servicios por procesos, funciones o interacciones y el acceso a interfaces por roles, componentes o colaboraciones.&  
		
		\raisebox{-\totalheight}{\includegraphics[width=0.3\textwidth]{arquitectura_diseno/imgs/ADMre4.png} }
		\\ \hline		
		
		Realización & 
		
		
		La relación de realización vincula una entidad lógica con una entidad más concreta que la realiza..&  
		
		\raisebox{-\totalheight}{\includegraphics[width=0.3\textwidth]{arquitectura_diseno/imgs/ADMre5.jpg} }
		\\ \hline		
		
		Asignación & 
		
		La relación de asignación vincula unidades de comportamiento con elementos activos (por ejemplo, roles, componentes) que los realizan o roles con actores que los cumplen.&  
		
		\raisebox{-\totalheight}{\includegraphics[width=0.3\textwidth]{arquitectura_diseno/imgs/ADMre6.png} }
		\\ \hline		
		
		Agregación & 
		
		
		La relación de agregación indica que un objeto agrupa una cantidad de otros objetos.&  
		
		\raisebox{-\totalheight}{\includegraphics[width=0.3\textwidth]{arquitectura_diseno/imgs/ADMre7.png} }
		\\ \hline		
		
		Composición & 
		
		
		La relación de composición indica que un objeto está compuesto por uno o más objetos diferentes.&  
		
		\raisebox{-\totalheight}{\includegraphics[width=0.3\textwidth]{arquitectura_diseno/imgs/ADMre8.png} }
		\\ \hline		
		
	\end{tabular}
	\begin{tabular}{|c | p{5cm} |  p{5cm}|}
		\hline
		Relaciones dinamicas && Notación
		\\ \hline
		
		Flujo & 
		La relación de flujo describe el intercambio o transferencia de, por ejemplo, información o valor entre procesos, función,
		interacciones y eventos.&  
		
		\raisebox{-\totalheight}{\includegraphics[width=0.3\textwidth]{arquitectura_diseno/imgs/ADMre9.png} }
		\\ \hline		
		
		Disparo & 
		
		La relación desencadenante describe las relaciones temporales o causales entre procesos, funciones, interacciones y eventos.&  
		
		\raisebox{-\totalheight}{\includegraphics[width=0.3\textwidth]{arquitectura_diseno/imgs/ADMre10.png} }
		\\ \hline		
	\end{tabular}
	
	
	\begin{tabular}{|c | p{5cm} |  p{5cm}|}
		\hline
		Otras relaciones && Notación
		\\ \hline
		
		Agrupación & 
		
		La relación de agrupamiento indica que los objetos, del mismo tipo o diferentes tipos, pertenecen juntos en función de alguna característica común.&  
		
		\raisebox{-\totalheight}{\includegraphics[width=0.3\textwidth]{arquitectura_diseno/imgs/ADMre11.png} }
		\\ \hline		
		
		Unión & 
		
		Un cruce se usa para conectar relaciones del mismo tipo.&  
		
		\raisebox{-\totalheight}{\includegraphics[width=0.1\textwidth]{arquitectura_diseno/imgs/ADMre12.png} }
		\\ \hline		
		Especialización & 
		
		La relación de especialización indica que un objeto es una especialización de otro objeto.&  
		
		\raisebox{-\totalheight}{\includegraphics[width=0.1\textwidth]{arquitectura_diseno/imgs/ADMre13.jpg} }
		\\ \hline		
	\end{tabular}
	
\end{center}

\newpage
\section{Tabla de Motivación Migración}

\begin{center}
	\begin{tabular}{|c | p{5cm} |  p{5cm}|}
		\hline
		Concepto & Definición & Notación
		\\ \hline
		
		StakeHolder & 
		
		
		El rol de un individuo, equipo o
		organización (o clases de los mismos) que
		representa sus intereses o preocupaciones en relación con el resultado de la arquitectura..&  
		
		\raisebox{-\totalheight}{\includegraphics[width=0.3\textwidth]{arquitectura_diseno/imgs/ADMmo1.png} }
		\\ \hline		
		
		Driver & 
		
		Algo que crea, motiva y alimenta el cambio en una organización.&  
		
		\raisebox{-\totalheight}{\includegraphics[width=0.3\textwidth]{arquitectura_diseno/imgs/ADMmo2.jpg} }
		\\ \hline		
		
		Assesment & 
		
		El resultado de algún análisis de algún controlador.&  
		
		\raisebox{-\totalheight}{\includegraphics[width=0.3\textwidth]{arquitectura_diseno/imgs/ADMmo3.jpg} }
		\\ \hline		
		
		Meta & 
		
		Un estado final que una parte interesada intenta lograr.&  
		
		\raisebox{-\totalheight}{\includegraphics[width=0.3\textwidth]{arquitectura_diseno/imgs/ADMmo4.jpg} }
		\\ \hline		
		
		Requerimiento & 
		
		Una declaración de necesidad que debe ser realizada por un sistema.&  
		
		\raisebox{-\totalheight}{\includegraphics[width=0.4\textwidth]{arquitectura_diseno/imgs/ADMmo5.jpg} }
		\\ \hline		
		
		Restricción & 
		
		Una restricción en la forma en que se realiza un sistema.&  
		
		\raisebox{-\totalheight}{\includegraphics[width=0.4\textwidth]{arquitectura_diseno/imgs/ADMmo6.jpg} }
		\\ \hline		
		
		Principio & 
		
		
		Una propiedad normativa de todos los sistemas en un contexto dado, o la forma en que se realizan.&  
		
		\raisebox{-\totalheight}{\includegraphics[width=0.3 \textwidth]{arquitectura_diseno/imgs/ADMmo7.jpg} }
		\\ \hline		
	\end{tabular}
\end{center}

\newpage


\section{ADM}
contenido...
\begin{figure}[th!]
	\centering
	\includegraphics[width=0.7\linewidth]{arquitectura_diseno/imgs/adm}
	\caption{ADM}
\end{figure}
