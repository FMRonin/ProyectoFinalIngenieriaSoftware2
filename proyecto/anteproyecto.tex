\chapter{Proyecto}
\section{Introducción}
La arquitectura de software es hoy en día una de las ramas de la ingeniera de software que toma cada vez más fuerza por ser primordial en las primeras etapas del desarrollo del software y en general en todo el proceso. Aun así hay muchos sistemas en especial a nivel web que presentan dificultades y que no se han desarrollado de la manera adecuada. Por otro lado, el auge de la web a aumentado el acceso a esta a personas en todo el mundo y a llevado a que las organizaciones inviertan un gran capital en mejorar su presencia y su imagen en el medio virtual. Ahí surge la necesidad de desarrollar gestores de contenido que faciliten los procesos publicación y modificación de información que las compañías presentan en la web. Este trabajo pretende generar una propuesta arquitectónica para un gestor de contenidos a partir de un caso de estudio (la organización IEEE Sección Colombia). Para ello se va a evaluar el estado actual del sitio web, las actividades que desempeña la organización y el contenido que esta requiere manejara en su sitio. También se va a hacer un proceso de recolección de requerimientos, con estos requerimientos, Se diseñará una arquitectura web a partir del modelo 4+1 que permite visualizar la propuesta en todos los niveles. Finalmente, se junto con la arquitectura propuesta se hará una evaluación del proceso de migración e implementación y generar toda la documentación necesaria para su implementación y mantenimiento.

\section{TITULO Y DEFINICIÓN DEL TEMA DE INVESTIGACIÓN}

Este trabajo pretende presentar una propuesta arquitectónica aplicada a la gestión de contenido y aplicada a un caso de estudio. Se evaluará desde el proceso de captura de requerimientos, análisis de estos y el posterior diseño de la arquitectura. También pretende documentarse de los patrones de arquitectura para permitir la definición y evaluación del modelo propuesto. De esta manera se puede definir el título del proyecto como:


Titulo: Diseño de una arquitectura para la gestión de contenidos a partir del uso de la selección y uso patrones de arquitectura: caso de estudio


\section{ESTUDIO DEL PROBLEMA DE INVESTIGACIÓN}

\subsection[short title]{title}
\newpage
